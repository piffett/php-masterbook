\chapter*{概要}

この本はProgateやPaizaやdotinstallをやって、なんとなくフレームワークとかも使えるようにはなったけど、フレームワークが何しているのかわからない!というような人に向けて、Webアプリを作りながら少しずつWebに関係する様々な技術や考え方を学んでいきます。
 
\section*{なぜこの本を書くのか}

昨今のプログラミングブームによって、プログラミング言語やフレームワークを道具として使うための教材は豊富に揃ってきました。しかし、2000年代と違いWeb開発の難易度は大きく上がりました。Laravelのような重厚なフレームワークが台頭し、AWSを始めとするクラウドサービスも普及し、これらのサービスを使いこなすことが必要不可欠になりました。また、GitやDockerのようなツールを使えることがWebエンジニアのリテラシーとなり、より一層Web開発の基礎知識を学ぶことが難しくなってきました。この教材は最初に生PHPでアプリケーションを作りながら徐々に抽象化していって、小さなフレームワークを作っていきます。その過程の中で、現在のフレームワークがどのような思想で作られているかを学びます。

\section*{PHPで学ぶ理由}

PHPはWeb開発を行うのに特化した言語です。多くのプログラミング言語の場合、小さなWebアプリケーションを開発する場合でも複数のライブラリを導入する必要がありますが、PHPの場合は必要ありません。また、徳丸本など、Webに関連する技術をPHPで説明するケースも多いため、PHPを採用しています。

